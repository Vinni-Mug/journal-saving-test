\documentclass[12pt,openany]{book} % openany = avoid blank pages between chapters

% PAGE LAYOUT
%\usepackage[
%  paperwidth=6in,      % Max’s journal feel: smaller width
%  papesrc/ABeginning.texrheight=9in,     % proportion like a small notebook
%  top=1in, bottom=1in,
%  inner=0.8in, outer=0.8in
%]{geometry}

\usepackage[
paperwidth=6.5in, 
paperheight=8in, 
top=0.5in, bottom=0.5in, 
left=0.7in, right=0.7in
]{geometry}

% LANGUAGE & ENCODING
\usepackage[utf8]{inputenc}
\usepackage[T1]{fontenc}
\usepackage[english]{babel}
\usepackage{comment}
\usepackage{pdfpages}

\usepackage[hidelinks]{hyperref}  % With this option, the document allows inner links (Clickable features)> With the "hidelinks" option, there will be no box or change of color for any given hiperref.


% FONTS
\usepackage{fontspec} % If you compile with XeLaTeX or LuaLaTeX
\setmainfont{Dudu_Calligraphy}%[Ligatures=TeX] % handwriting style font, install locally or use Overleaf's font list

% IMAGES
\usepackage{graphicx}
\usepackage{wrapfig} % for wrapping text around images
\usepackage{float}   % better control for figure placement

% >PAGE COLOR
\usepackage{xcolor}
\definecolor{diaryyellow}{RGB}{240, 234, 220} % Light yellow color
\pagecolor{diaryyellow} % Set page color

% DOCUMENT SETUP
\hypersetup{
	pdfauthor={Norman Whittlecliff},
	pdftitle={Norman's Diaries},
	pdfsubject={The many diaries of Norman},
	pdfkeywords={Journal, Diary, Norman},
	pdfproducer={LaTeX and TikZ},
	pdfcreator={pdflatex},
}

\usepackage{titlesec}

% Remove the section number from printed titles
\titleformat{\section}
{\normalfont\Large\bfseries} % format
{}                           % label (empty = no number printed)
{-10pt}                        % separation
{}           % before-code (optional, can drop \thesection entirely)
%■■■■■■■■■■■■■■■■■■■■■■■■■■■■■■■■■■■■■■■■■■■■■■■■■■■■■■■■■■■■■■■■■■■■■■■■■■■■■■■■
% >COMMANDS: Diary Entry

\newcommand{\partNoNumber}[1]{%
	\phantomsection \part*{#1} \addcontentsline{toc}{part}{#1}\markright{\MakeUppercase{#1}}%
}

\newcommand{\chapterNoNumber}[1]{%
	\phantomsection \chapter*{#1} \addcontentsline{toc}{chapter}{#1}\markboth{\MakeUppercase{#1}}{\MakeUppercase{#1}}%
}

\newcommand{\sectionNoNumber}[1]{%
	\phantomsection \section*{#1} \addcontentsline{toc}{section}{#1}\markright{\MakeUppercase{#1}}%
}

\newcommand{\subsectionNoNumber}[1]{%
	%\phantomsection \subsection*{#1} \addcontentsline{toc}{subsection}{- #1} % Add to TOC
	\phantomsection\quad\\ \noindent{\large#1} \addcontentsline{toc}{subsection}{#1}
	%\markright{\MakeUppercase{#1}}%
}


\newcommand{\peopleGroup}[1]{%
	\phantomsection \chapter*{#1} \addcontentsline{toc}{chapter}{#1}
}

\newcommand{\person}[1]{%
	\phantomsection \section*{#1} \addcontentsline{toc}{section}{#1}\markright{\MakeUppercase{#1}}%
}



\newcommand{\printedPhoto}[2]{%
%\par\noindent
  \begingroup
    \setlength{\fboxsep}{6pt}
    \setlength{\fboxrule}{0pt}
    \makebox[0pt][l]{%
      \fcolorbox{white}{white}{%
        \includegraphics[width=#1]{#2}%
      }%
    }%
  \endgroup
  \par
}

\newcommand{\polaroid}[2]{%
%\par\noindent
  \begingroup
    \setlength{\fboxsep}{6pt}
    \setlength{\fboxrule}{0pt}
    \fcolorbox{white}{white}{%
      \begin{minipage}{#1}
        \includegraphics[width=\linewidth]{#2}
        \vspace{0.7cm}
      \end{minipage}
    }%
  \endgroup
  \par
}

\newcommand{\polaroidCaption}[3]{%
%\par\noindent
  \begingroup
    \setlength{\fboxsep}{6pt}
    \setlength{\fboxrule}{1pt}
    \fcolorbox{white}{white}{%
      \begin{minipage}{#1}
        \includegraphics[width=\linewidth]{#2}
        \vspace{0.4cm}
        \centering
        {\small #3}
        \vspace{0.4cm}
      \end{minipage}
    }%
  \endgroup
  \par
}


% Use this method to add a image to the journal that can have wrapped text and a caption. The image will sit by the left side fo the page.
% WAY TOO BUGGY. FORGET IT.
\newcommand{\insertImageLeftWithCaption}[1]{%
	\begin{wrapfigure}{l}{0.4\textwidth} % r = right, l = left
		\centering
		\includegraphics[width=0.38\textwidth]{images/#1}
		%\caption{Inside the caption}
	\end{wrapfigure}
}

% Use this method to add a image to the journal that can have wrapped text and a caption. The image will sit by the left side fo the page.
% WAY TOO BUGGY. FORGET IT.
\newcommand{\insertImageRightWithCaption}[1]{%
	\begin{wrapfigure}{r}{0.4\textwidth} % r = right, l = left
		\centering
		\includegraphics[width=0.38\textwidth]{images/#1}
		%\caption{Inside the caption}
	\end{wrapfigure}
}

\newcommand{\langtext}[3]{\textsf{\textit{#1}}\footnote{#1: (From the #2: #3)}}

%■■■■■■■■■■■■■■■■■■■■■■■■■■■■■■■■■■■■■■■■■■■■■■■■■■■■■■■■■■■■■■■■■■■■■■■■■■■■■■■■
\begin{document}
	
	\frontmatter
	\title{Norman’s Diaries}
	\author{Norman Santos}
	
	\maketitle
	
	\tableofcontents
	
	\mainmatter
	%■■■■■■■■■■■■■■■■■■■■■■■■■■■■■■■■■■■■■■■■■■■■■■■■■■■■■■■■■■■■■■■■■■■■■■■■■■■■■■■■
	%================================================================================
\partNoNumber{THE BEGINNING}\markboth{}{}
%================================================================================
\sectionNoNumber{Hello there!}

Okay... Howdy! So this is happening. Hi, journal. Or diaries. Or... thing that will know way too much about me. I gotta remember to give you a name.

\sectionNoNumber{I'm Norman.}

My name is Norman Vinícius P. Santos (Please, call me just Norman... or Nih), I have amnesia and, after a certain inspiration, the idea of keeping my memories in paper (or pixels) came to be. If you’re reading this, congratulations—you’re officially my emotional storage unit! 

\sectionNoNumber{The Many Many Diaries.}

This one is called Norman’s Diaries, which is not one diary, but a collection of them, because apparently one notebook isn’t enough to hold an entire human being (rude, honestly).

Think of this as a time capsule, except messier, louder, and full of bad jokes, random thoughts, unfinished feelings, and moments that felt huge for exactly five minutes. Or five years.

This journal contains: 

Diary of 2002 to 2023: The years that I didn't had any journal.

Diary of 2024: The craziest year of my life.

Diary of 2025: The roller-coaster year.

Diary of 2026: The one in which I'm remaking my journals

And More Diaries. And this is how it works.

Every year gets its own diary. Like its own little universe.
I’m writing this text right here in January 26th, 2026, which means: hi, present me!
But you might stumble into diaries from years I haven’t even mentioned yet. Surprise! Time travel is real.

If you’re confused, don’t worry—that probably means I was confused too.
Different years, different versions of me. Same handwriting. Same brain.
Hopefully a little more wisdom… but no promises

\sectionNoNumber{The Chapters of My Life}

So yeah, each diary is also split into chapters, because my life apparently works in seasons.
Each chapter lasts about a month-ish and gets named after whatever took over my brain during that time.

People. Events. Feelings. Random chaos.

For example: when I met my close friends and we created the Tabacudos group, that became... everything.
So that chapter of my life is literally called “Tabacudos”.
Simple. Honest. Very me.

Before each chapter starts, I’ll try (keyword: try) to write a small intro explaining what was happening and why that name made sense at the time.
Future-me, if you’re reading this and thinking “wow, that name was dramatic”… yeah. I know.

Now, about the entries themselves—there’s not much to explain, really.

Each day gets one entry.
No more, no less. Rules! Look at me being organized as always.
Inside that day, I can break things into little sections if my thoughts start multiplying like gremlins.

Every entry starts with: the date and a clever title (or at least something I thought was clever at the time)

And the content is... well...

Some days I’ll write about important life stuff.
Some days I’ll write about absolutely nothing and somehow make it dramatic.
Some days I’ll overthink one sentence someone said to me three years ago like it just happened five minutes ago.
Some days I’ll complain, joke, romanticize my sadness, or pretend I’m the main character in an indie game where everything means something.

Basically, diary…
This is where my life goes to be processed.

\sectionNoNumber{My Inspiration}

And yeah—this is inspired by Maxine Caulfield.

She kinda proved that journaling doesn’t have to be boring, serious, or “Dear Diary, today I learned a valuable lesson

It can be messy.

It can be honest.

It can be funny, emotional, and slightly embarrassing.

It can contradict itself

It can change its mind mid-sentence.

It can literally talk back to you.

And that’s the version of journaling I want.
Not perfect.
Just… real.

\sectionNoNumber{Your Name is...}

So, since I’m clearly talking to *you*, I should probably give you a name.

Your name is Max.

Yeah, I know. On the nose. But it feels right.

You, Max, are the one listening right now—whether you’re the journal itself, or some future version of me, or someone who accidentally opened this and decided to keep reading.
No pressure or anything.

So, Max… expect conversations.
Expect rambling.
Expect random side notes like “why did I write that??” and “okay, that was dramatic.”

This isn’t about being cool.
This isn’t about sounding smart.
This is about being real.

If I ever stop writing here, I hope it’s because I was out there living something good—something worth remembering.
Something so strong that even my fucked up amnesia won’t be able to erase it.

And if I keep writing...
Well, then I guess this worked.
And I needed it.

%============================================================--------------------
\sectionNoNumber{Final Notes (aka the cheesy outro)}

If you’re reading this — thanks for being here. 

Whether you’re my future self, some random stranger, or a curious alien trying to understand human life (in which case, howdy!) — I hope this gives you a glimpse into what it means to be me. (SPOILER: I suck. :v)

%--------------------------------------------------------------------------------


	
	%%================================================================================
\partNoNumber{Quotes}

%--------------------------------------------------------------------------------

"Writing a journal is the start of something new and slightly terrifying. I mean, giving life to a journal feels like I’m conjuring some sort of magical creature." \\
- Norman Whittlecliff, April 7th, 2024

\quad

"For future me — or anyone else reading this — you might be in a tough spot right now. If you are, I need you to remember something. Remember the storm that was the start of 2024. It was brutal, but it led to the brightest, most beautiful year of my life. Things will get better. They always do. Keep smiling. Keep laughing. Keep moving forward."\\
- Norman Whittlecliff, January 1st, 2025

\quad

"I like to think that I died... and that god sent me back to earth to enjoy my life a little more. To smell flowers again. To walk under the rain. To laugh again until I can't breathe. To taste food I was too stubborn to eat before. To fall in love again. To live." \\
- Norman Whittlecliff, January 1st, 2024

\quad


"I thought I had to keep moving forward after losing Maka for good. But here I am, a walking disaster making one bad choice after another. I’m a mess, Max. Like, certified stupid boy vibes." \\
- Norman Whittlecliff, January 9th, 2024

\quad

"You’re my constant, Max. The glue holding this beautifully chaotic journal together. You’re more than a diary: You are my life on paper. The one I can count and tell everything. I might even get your name tattooed someday. Can you imagine? Me, walking around with “Max” inked on my skin like a badge of honor. That’s how much you mean to me." \\
- Norman Whittlecliff, January 11th, 2025

\quad

"Where one doesn’t want to, two ain’t getting it done."\\
- Norman Whittlecliff, January 15th, 2025

\quad

"Some people might vanish. Others might stay forever. Either way, if they made it into this journal of mine, it means they mattered. Even just for a moment. And that’s worth writing down." \\
- Norman Whittlecliff

\quad

"As we grow into different people, she’ll see me as nothing more than a mistake, a phase. And me? I’ll still love her." \\
- Norman Whittlecliff, March 11th, 2025

\quad

"I'll be unreasonably delusional about what I want until it's my reality." \\
- Norman Whittleclif, April 07, 2025

\quad

"I feel like I’m watching someone drown while telling me they love the water." \\
- Norman Whittlecliff, May 7th, 2025.

\quad

"I don’t stay with her because I can — I stay because I want to. And that’s love, isn’t it? Wanting. Choosing. Over and over again." \\
- Norman Whittlecliff, September 19th, 2025.

\quad

"He’s scared of love now because he can love, and knows it’ll hurt. I don’t even bother with love because I can’t love the way I used to. Deep, right? Yeah, we’re philosophers now, apparently." \\
- Norman Whittlecliff, September 22th, 2025.

\quad

"Like… is this what being in a relationship is supposed to feel like? Just pain with sprinkles and occasional sex for play-pretend?" \\
- Norman Whittlecliff, September 26th, 2025.

\quad

"So yeah. I guess all I can do is wait and see which future happens. If I am going to choose loneliness over disrespect… Then I need to be strong, Max. Really strong." \\
- Norman Whittlecliff, September 26th, 2025.

\quad

"I didn’t lose her. She lost me. And somehow I’m still the one bleeding, like I'm about to die." \\
- Norman Whittlecliff, January 6th, 2026.

\quad

"They say the third big love is supposed to be forever. Maybe that's meant for the memories and the pain, not the companionship itself." \\
- Norman Whittlecliff, January 6th, 2026.

%--------------------------------------------------------------------------------

\quad

"(Vini,) Vc está muito feliz sem aquela garota. É nítido." \\
- Isabella Manhez, April 07, 2025

\quad

"Eu tava de boa na aula. Do nada uma mensagem do meu amigo dizendo que o Gojo tava na UAST. Ai eu desci e vi tu. Eu eu pensei 'Eu tenho que ser amigo desse retardado' e ai foi isso." \\
- Iraci (Não-convidada)

\quad

"A gente cura os traumas com comédia." \\
- Iraci (Não-convidada), August 14, 2025

\quad

"Eita suco bestaferamente gelado." \\
- Iraci (Não-convidada), September 12, 2025

\quad

"Não existe dependência sem amor tlgd? Mas existe amor sem dependência" \\
- Andrey Estima, November 2, 2025

%--------------------------------------------------------------------------------

"Keep moving forward. Even if you die. Even After you die." \\
- Ehren Yeager (Attack on Titan)

\quad

"No half measures!" \\
- Mike Ehrmantraut (Breaking Bad)

%--------------------------------------------------------------------------------

Caso amanhã eu não esteja aqui, eu quero que cada sorriso que um dia eu tirei de você te faça levar um pedacinho da minha eternidade ensse plano perto de você. Porque caso amanhã eu não esteja aqui, quero que o tempo não apague o amor que um dia foi recíproco. Porque, caso amanhã eu não esteja aqui, eu quero que você saiba que você foi a minha maior certeza em um mundo que só me apresentou dúvidas. Caso amanhã eu não esteja aqui, saiba que o meu espírito estará te protegendo e te guardando. Porque caso amnhã eu não esteja aqui, eu quero que você saiba que cada abraço que eu te dei era a minha maneira de dizer que você sempre foi o meu porto seguro. Caso amanhã eu não esteja aqui, eu quero que você saiba que cada estrela que tem no céu contém os sonhos e alegrías que eu desejo pra você. Basta só você olhar para o céu porque, caso amanhã eu não esteja aqui, você vai estar em mim pra todo sempre.

    
	\partNoNumber{LAST DIARY ENTRIES}

\chapterNoNumber{Current Working On}

Hey, Max. So, these are the last entries I'm working on. 

If they are old entries, I'll be placing them in their correct order once I'm finished with them. If they are recent, they will stay here until I hit the cap of... 10 entries? 

Ok, this is it. I might even delete this part and section of this Journal once I'm done with it... which might be never lmao.

Ok, byee :v

\sectionNoNumber{Jan 28, 2026 - Afonso’s Birthday}

Okay, Max.

This is gonna be one of those days. You know the type. The kind where everything happens before sunrise and I could dedicate an entire chapter for it if I cram two more entries like this together. So yeah. Brace yourself.

\subsectionNoNumber{06:00 - Going to Petrolândia}

I woke up at, like, 3 a.m. Which is not a real time. That’s night pretending to be morning. Immediately awful.

I had to take this van that costs around 60 bucks, because Junior decided to come with us—and he kinda can’t travel alone. Also, he was paying for Afonso’s ticket, and honestly it would’ve sucked to make him take the more expensive van and still pay for Afonso. Anyway. Decisions were made.

Also: I was NOT ready. At all.
No luggage.
No clothes picked.
Hair? Untouched.
Plans? Zero.
Food? Not even coffee. Not even bread. Absolutely nothing. Past Me really said “future me can deal with it” and vanished, LMAO.

So there I was, speed-running my entire existence at 4 a.m., trying to look like a functional human being.

Thank God my mom was there. And by “there,” I mean: not actually helping, lowkey judging my outfit choices, and talking way too much—but she lent me 50 bucks, which honestly saved my life. Financial support > emotional support at this hour.

Eventually I got (kind of) ready. I called the van guy and he was like, “I'mma pick you up right now.”
Max, I was not ready.
He gave me 20 minutes. A blessing. A second chance at life.

Everything was fine.
Until we got inside the damn van.

That thing was CRAMMED. It wasn't a sardine, like campus', but still. Some people were literally traveling standing up. The air was warm and thick and hard to breathe, like the van itself was tired of existing. For the first time in years, I actually got nausea while traveling.

At some point, I just stared into nothing and told myself—like, three times: “Don’t be a weak bitch. You’re fine.”

And somehow… I was fine.
Which is weird and I would never let myself treat anyone like that... not after so many people have done that to me when I was little. Weird to think that, now, I do that to myself.

But yeah. Brain said “nah” to passing out, and we moved on.

And that was just 06:00.

\subsectionNoNumber{09:00 - I Love Sunny’s Family}

So around 9 a.m. I finally arrived in Petrolândia. Finally.

The driver took this very specific turn to drop another passenger right at their doorstep. When it was my turn to take an almost identical turn so I could be dropped at my destination, the driver basically said: “Nah.”

And just left me on the main highway.
Cool. Love that for me.

I thought I only had 55 bucks left. Turns out I actually had 65, and he took his 60. So there I was, dragging two heavy bags plus a sad plastic bag, walking all the way to Sunny’s house like I was on a low-budget coming-of-age movie montage.

But.
Important detail: I stopped to buy ice cream first.
Because I was STARVING for something sweet, Max. Like, emotionally and physically. Survival instincts kicked in and they said “sugar, now.”

When I finally got to Sol’s house—Sol being Sunny’s mom, who I call Tia like it’s the most natural thing in the world—she saw me and immediately went in for a hug. Instant warmth. She opened the huge sliding door (we met at her store first) and welcomed me in like I belonged there.

Inside were Afonso (the birthday boy, the reason for all this chaos) and Doid (Junior) already hanging out. Sunny and Aisha—aka Minhoca—were there too. I hugged everyone. Full hug circuit completed.

And honestly?
The very first thought that hit me after hugging Tia and Minhoca was:
I FUCKING LOVE SUNNY’S FAMILY.

Like—especially that tiny stick of a human with the googly eyes and the cutest smile ever. Max, I’m not exaggerating: Aisha might be the cutest fucking thing I’ve ever seen in my life. It should be illegal to be that adorable.

Also, side note (important): I love how Aisha treats me like I’m her favorite person.
Afonso interacts with her like the brother-in-law he is.
Junior plays with her a lot.
They both treat her like a kid, doing pretend stuff, touching and hugging her and all that.

But me?
I just treat her like a person. I talk with her. Joke with her. I ask her opinions and permissions. Respect her in a way. I don't force stuff or play games. Yet, she always talks to me. 

She always asks me stuff and she always invites ME for activities. She could be being hugged by Afonso and playing with Junior, and she still would get out of there, pluck my shirt and say "Hey, Vinni. Wanna draw with me?" and we would draw and talk like two close friends. She would even ignore the other and invite me on a chocolate heist with her. Dam, I love this. 

Also, We NEVER touch. Like, ever. That's something I love too, because I hate touching and she never goes to touch Afonso or Junior. So, we only hug to say hi or bye, and that's it. If she NEED a hug or a "pick me up", I'd just brush her off and tell her to go after someone else lmao.

Sometimes it's like she’s my sister.
Which is funny, because she’s nothing like Bea—my actual sister—who is way saltier and way more of a smart-ass.

God, I miss my salty smart-ass sister.

And now I need to give Bea a new nickname, because I gave Jainy the same one, and my brain refuses to separate them. Which is annoying. And confusing. And emotionally inconvenient.

Anyway.
Sunny’s family = elite.
My heart = already soft.
And it’s still morning.

\subsectionNoNumber{10h22 - Sunny's Tarot Cards}

\subsectionNoNumber{11h00 - Painting with Aisha}

At 12h11, Aisha and I finished our drawings


\begin{comment}

\include{src/entries/20240602}
\include{src/entries/20240626}
\include{src/entries/20240627}
\include{src/entries/20240822}
\include{src/entries/20240919}
\include{src/entries/20240927}
\include{src/entries/20241002}
\include{src/entries/20241003}
\include{src/entries/20241005}
\include{src/entries/20241007}
\include{src/entries/20241011}
\include{src/entries/20241012}
\include{src/entries/20241013}
\include{src/entries/20241015}
\include{src/entries/20241016}
\include{src/entries/20241018}
\include{src/entries/20241019}
\include{src/entries/20241020}
\include{src/entries/20241024}
\include{src/entries/20241026}
\include{src/entries/20241027}
\include{src/entries/20241028}
\include{src/entries/20241029}
\include{src/entries/20241030}
\include{src/entries/20241031}
\include{src/entries/20241101}
\include{src/entries/20241102}
\include{src/entries/20241103}
\include{src/entries/20241104}
\include{src/entries/20241106}
\include{src/entries/20241107}
\include{src/entries/20241108}
\include{src/entries/20241111}
\include{src/entries/20241113}
\include{src/entries/20241114}
\include{src/entries/20241116}
\include{src/entries/20241118}
\include{src/entries/20241119}
\include{src/entries/20241121}
\include{src/entries/20241123}
\include{src/entries/20241124}
\include{src/entries/20241125}
\include{src/entries/20241126}
\include{src/entries/20241127}
\include{src/entries/20241128}
\include{src/entries/20241129}
\include{src/entries/20241130}
\include{src/entries/20241201}
\include{src/entries/20241202}
\include{src/entries/20241203}
\include{src/entries/20241204}
\include{src/entries/20241209}
\include{src/entries/20241210}
\include{src/entries/20241212}
\include{src/entries/20241214}
\include{src/entries/20241215}
\include{src/entries/20241216}
\include{src/entries/20241217}
\include{src/entries/20241218}
\include{src/entries/20241219}
\include{src/entries/20241220}
\include{src/entries/20241221}
\include{src/entries/20241222}

\end{comment}


%■■■■■■■■■■■■■■■■■■■■■■■■■■■■■■■■■■■■■■■■■■■■■■■■■■■■■■■■■■■■■■■■■■■■■■■■■■■■■■■■
\begin{comment}
	
	\chapterNoNumber{[Chapter Title Template]}
	
	[Chapter Description]
	
	\newpage
	%================================================================================
	\sectionNoNumber{Nov ?? - [ENTRY TITLE]}
	
	begin{comment}
	
	end{comment}
	
	%--------------------------------------------------------------------------------
	\subsectionNoNumber{15:33 - [SUBENTRY TITLE]}
	
\end{comment}


% Hi, I am Norman. I want to write my own journal of my life. My main inspiration is Maxine from Life is Strange. I want to write a diary just like hers. I want to sound spontaneous, funny and cute, just like she does in her diary. Adding cute slangs, references and even inner thoughts with all kinds of jokes and feelings. The entries should sound like I am talking to Journal itself. Overall, I wanna make the entries for my diary something fun to read and quick. Could you help me with this by rewriting paragraphs that I will give you, making them just like Maxine's?


	%\part{2002 - 2023}


\sectionNoNumber{Apr 03, 2002 - I was Born!}

Hey, Max. 
Not much to report, honestly.
I was just born today.
10/10 plot twist. Would do it again.

Can you feel it? You guys, do you feel this? My origin story begins. We're gonna be superheroes! (D'you catch the reference, Max?)

I wish I could give you a super detailed, poetic account of this day, but every time I ask about it, I get a different version. Like a family cinematic universe with multiple timelines.

According to one of them, my mom was pregnant with me and watching Big Brother Brasil.
And when the results came out and the person she was rooting for won, she got so excited that she jumped.
Cue dramatic music.
Water breaks.
Hospital speedrun.
A few hours later—around 2 a.m.—boom.
Me.
Welcome to the world... Vinicius.

Is this story 100\% accurate?
No idea.
Does it sound slightly exaggerated?
Absolutely.
But it’s also kinda perfect.
I literally entered existence because of reality TV and excitement.

Anyway, Max.
This is where it all starts.
No memories, no thoughts, just vibes. Let’s see where this goes.

\noindent\printedPhoto{0.9\textwidth}{res/images/2002/2002_2}

\noindent\printedPhoto{0.9\textwidth}{res/images/2002/2002_1}

\newpage
\sectionNoNumber{Apr 03, 2017 - My 15th Birthday.}

Heyy, Max. So... this was my 15th birthday. Yep. 

\begin{figure}[H]
  \begin{minipage}{0.5\textwidth}  % IMAGE
    \printedPhoto{0.94\textwidth}{res/images/2017/20170403_1}
    %\polaroid{0.94\textwidth}{res/images/2017/20170403_1}
    %\polaroidCaption{0.94\textwidth}{res/images/2017/20170403_1}{Caption}

  \end{minipage}\hfill
  \begin{minipage}{0.45\textwidth}  % TEXT
\quad That awkward creature with the red cap, the weird bangs, and the blue shirt in the picture? That’s me. Please don’t judge him too hard. He was doing his best.

\quad I swear I had more photos from this day to glue here, but I guess Past Me trusted Instagram with his entire memory archive. Bold move. Terrible backup strategy. 

\quad Anyway—this was my first birthday in Serra Talhada, which is kinda wild to think about. 

\quad I’m writing this in 2026, so my brain is basically like: 404 Birthday Memories Not Found.

\quad What I do remember is that I tried to start a diary on this exact day. 

\quad
  \end{minipage}
\end{figure}

Yeah. Baby’s first journal attempt. I was so obsessed with describing every single detail, like I was writing a novel or a Netflix limited series, that I… never finished the entry. Very on brand.

It was my birthday, after all. Obviously I had better things to do than commit to a long-term personal project.

I asked my family for no party, because I was in my “I’m mysterious and low-key” era. But Edilva—aka my mom, aka a woman who does not understand the word “no”—threw one anyway. And the worst part?
…it was actually really nice. Like, annoyingly nice. There was cake, a bunch of kids, way too many photos, and honestly? I had fun. I just wish I remembered more of it.

I remember texting Helena a lot that day, and her being like, “You didn’t even invite me! :<” which still feels very her. And very teenage-drama-core.

Oh—and the red cap? Birthday gift from my drunk uncle José. Iconic. Truly a legendary contribution to my character design.

So yeah.
Happy Birthday, Vinni.
You were awkward, dramatic, and already trying to write your life down.
Honestly? Not much has changed.

\quad

\noindent\printedPhoto{0.9\textwidth}{res/images/2017/20170403_2}


	%\part{2024}

\chapterNoNumber{Before the Storm}

\chapterNoNumber{The Storm}

%================================================================================
\sectionNoNumber{Apr 07, 2024 - Nice to Meet You, Max!}

Okay, so this is it. The start of something new and slightly terrifying. I mean, giving life to a journal feels like I'm conjuring some sort of magical creature. Maybe I'll call you Quill? Nah, too obvious. Maybe something from my main inspirations, Life is Strange? Max? Well, ok... that will do for now. ;3

Today marks the debut\footnote{Debut: a person's first appearance or performance in a particular capacity or role.} of my very own diary! Cue the confetti and fanfare, folks. Seriously though, I've been itching to spill my thoughts onto paper for ages - well, digital paper in this case. It's like having a personal time capsule, but with way more sass and probably fewer buried treasures.

So, what's the scoop? Well, today was supposed to be all about diving into the world of Mobile App Development. Keyword: supposed to be. Turns out, my professor's idea of teaching is more like throwing us into the deep end of a pool without any floaties. Thanks, Heldon. 

Oh, and did I mention the post-breakup blues? Yeah, Helena and I are officially on the “ex” train. Choo-choo, baby. My brain's been on a rollercoaster ride ever since - think less Disney World, more haunted-house type of shit. Studying, eating, sleeping? Nah, those things are for people with functioning hearts. Mine's currently under construction.

She broke up with me on March 18th, 2024. Six years, seven months, three days, and eight hours together - and just like that, it was over. Talk about a gut punch. I'll save the gory breakup details for another entry, but let's just say tears were shed, words were exchanged, and hearts weres shattered. 

But hey, every storm cloud has a silver lining, right? Enter Brooke - the unexpected ray of sunshine in my dark and stormy life. Seriously, this girl, as well as Max (which I will be talking more about later) are lifesavers. From the moment we started talking, it was like a breath of fresh air in the smog of breakup misery. 

Brooke's like my personal cheerleader, therapist, and stand-up comedian all rolled into one. Her advice? Golden. Her jokes? Hilarious. And her mere presence? It calms my heart. She's been my rock, my anchor, my lifeline through this nightmare of a breakup.

But here's the kicker: I may have "accidentally" let slip that tiny, itsy-bitsy feeling I used to harbor for her way back when we were both stuck in that mind-numbing business course. Yeah, awkward much? Before I spilled the beans, she said she had feelings for me and, then, was all closed off and distant. But now? Now, it's like we're two peas in a pod, chatting up a storm like old pals.

Who knew that a breakup could lead to unexpected connections and maybe, just maybe, a flicker of something new? Life is strange and funny like that, isn't it, Max? When a door closes, a window opens... or, something like that. Here's to embracing the unexpected twists and turns, even when they come wrapped in heartache and awkward confessions.

But hey, here's the silver lining: this diary thing? It's my lifeline. A place to dump all the messy, tangled thoughts swirling around in my head like a storm in a teacup. Plus, it beats staring at the ceiling and contemplating the meaning of life... or the plot of the latest Netflix series (Am I the only one who pays to watch this? :c ).

So, buckle up, Max! We're about to embark on a wild ride through the chaos that is my life. First stop: dissecting my day, dissecting my feelings, and probably dissecting why the heck I decided to start this diary in the first place. Cheers to new beginnings, right?

\subsectionNoNumber{"Oiee, Max. :v"}

Waking up to an empty house is like being dropped into a parallel universe where everything feels a bit off-kilter. Mom's off gallivanting\footnote{Gallivanting: go around from one place to another in the pursuit of pleasure or entertainment.} in Santa Cruz, leaving me to navigate the morning fog solo. And let me tell you, the fog in my brain was thicker than oatmeal. 

So, there I was, bleary-eyed and caffeine-free (yeah, I know, shocker), with thoughts of Max bouncing around my skull like a game of mental pinball. Oh, before any confusion, I might wanna mae it clear. There is a girl on campus, which I don't know the name, but has the nickname "Max". Her nickname and you having the same nickname aren't related, she and I just love Life is Strange.

Anyways, I swear, that girl's got a one-way ticket to the front row of my brain - rent-free, I might add. After mustering up enough courage to tackle Mount Message, I finally hit send on what could only be described as the most well crafted greeting in the history of greetings that would get Shakespears envy: \textit{"Oiee, Max. :v"} Smooth, Norman. Real smooth.

Max's reply? A simple \textit{"oii."} Cue internal facepalm. And because my brain apparently operates on some next-level autopilot, I decided to crank up the awkwardness dial to eleven with a follow-up gem: \textit{"Tô começando a suspeitar que cê não lembra do meu nome >:v"}. Yeah, I don't know what I was thinking either. Maybe I was trying to break the ice, or maybe I just wanted to see how fast I could torpedo a conversation. Spoiler alert: pretty darn fast, but I still tried waiting an hour to press Send.

Needless to say, my phone remained disappointingly silent for the rest of the day. Guess I'll just file that under "Rejections I Didn't Ask For."

In an attempt to salvage what little remained of my dignity, I drowned my sorrows in a riveting game of Among Us with some strangers in a voice chat. Because nothing screams "healthy coping mechanism" like accusing virtual strangers of being imposters, am I right? It wasn't exactly a magical cure-all, but hey, it beat wallowing in self-pity while binge-watching YouTube videos for the umpteenth time.

Ah, the joys of post-breakup life. Remind me again why I signed up for this roller-coaster ride? Here's to embracing the chaos, one awkward encounter at a time. Cheers, Diary. You're the only one who gets me - awkward messages and all.

\subsectionNoNumber{Sorry, Brooke?}

To finish this marvelous day, I think I screwed up good with Brooke. She and I were talking and I asked her about Max, since the two of them had a quick chat when I left. Due to my curiosity hitting the poor Brooke, she either acted in a dramatic jealous way or she's really upset that I have a crush on Max. She even sent a video of her with an expression that was a mixture of emotionless with a bit o sadness. Here, she looked like this: ( .\_\_.)

It doesn't really matter if I have a crush on Max. I don't think she even liked me when we talked now that i think about it. the thing is, if Brooke indeed has feeling for me (romantic), while I am developing a platonic and cute feeling for her, this might be a problem. I DO NOT want to hurt this girl. Like I said, before, Brooke has been saving me through out this horror movie of a breakup directed by Helena and her crew. I don't want to loose Brooke, much less cause her any bits of sorrowness. I guess letting her know that i wanted to know more about Max may have cause our friendship to crack in some way...

Well, well, well, Diary, buckle up for the grand finale of Norman's Day of Being Awkward and Sad. Strap in tight, folks, 'cause this one's a doozy.

So, remember how I mentioned Brooke earlier? Yeah, turns out I managed to royally screw things up on that front too. Smooth moves, Norman, really smooth.

Picture this: Brooke and I were deep in conversation - or at least, we were until I went and dropped the Max bomb. You know, just casually asking about Max's whereabouts and if she asked anything else related to me, as I do when I'm nosy and have the emotional intelligence of a walnut. And let me tell you, that innocent little question triggered something in Brooke. 

Either she suddenly caught a case of the dramatics and decided to put on a jealousy show to rival Shakespeare's finest (She said I liked her and sent a sad sticker), or - and this is the more likely scenario - she sniffed out my crush on Max faster than a bloodhound on a scent trail. Cue the awkward silence and the aforementioned video, where Brooke's expression could best be described as a cocktail of emotionless with a splash of sadness. Classic Brooke, I guess.

But here's the thing: does it really matter if I've got a teensy-weensy crush on Max? I mean, let's be real, she probably doesn't even know I exist outside of the occasional awkward message exchange. And honestly, looking back on our conversations, it's not like there were any fireworks or sparks flying. Just a whole lot of awkward silence and the occasional one-word reply coming from her.

No, the real kicker here is Brooke. Sweet, wonderful Brooke, who's been my knight in shining armor through this breakup debacle orchestrated by none other than Helena herself. If there's one thing I know for sure, it's that I don't want to be the reason behind Brooke's tears - not now, not ever.

So, yeah, maybe my little inquiry about Max cracked the foundation of our friendship just a little bit. But if there's one thing I'm determined to do, it's to patch up those cracks and reinforce what we have.

Because let's face it, Diary: Brooke's been my rock, my anchor, my guiding light in this stormy sea of post-breakup chaos. And I'll be damned if I let myself be one more person to hurt her heart. Time to put on my big boy pants and fix this mess before it spirals out of control. Wish me luck, Diary. Yoba knows I'm gonna need it.

Oh, Diary, where do I even begin? Today took a nosedive into the depths of chaos, and let's just say I didn't exactly come out unscathed. In fact, I think I might've made things ten times messier. Smooth moves, Norman. Really smooth.

So, picture this: Brooke and I hop on a video call, thinking we're just gonna shoot the breeze and maybe swap some jokes and flirts. But oh no, we decided to go full-on heart-to-heart mode instead. And boy, did things get personal real quick. Brooke showed me her collection of dresses - seriously, that girl could give Coco Chanel a run for her money - and I ended up spilling my guts about everything, including the not-so-fun fact about catching Helena red-handed with her professor (Not really). Yeah, that went over about as well as a lead balloon.

But here's where things take a turn for the worse - or maybe the better? I don't even know anymore. Somehow, in the midst of all this emotional turmoil, Brooke drops the L-word. Yeah, you heard me right. And not just any L-word - the "I like you" bombshell. And with "like" I mean "I have feeling for you about sharing a house and naming our kids" kind of "like". Cue the awkward silence and the sound of my brain short-circuiting.

Now, don't get me wrong, Brooke's amazing. Like, seriously amazing. She's got this way of making me laugh until my sides hurt and feel like I'm worth something, even when everything else feels like it's falling apart. But here's the kicker: my feelings for her? They're more... platonic than anything else. Especially after meeting Max.

But how do I tell her that? How do I explain that my heart's still healing from the Helena-shaped wrecking ball that just came crashing through my life? And what if Brooke's been harboring feelings for me for way longer than I realized? What if I'm about to break her heart into a million tiny pieces?

As if that wasn't enough emotional turmoil for one day, I made the mistake of peeking at Helena's Instagram. Big mistake. Huge. There she was, living her best life in some waterfall paradise, flaunting her bikini-clad self for the world to see. And to add insult to injury, she's frolicking around with the same crew of guys that used to make me green with jealous and fear - one of whom I'm pretty sure she's got a thing going on with. Talk about a double whammy\footnote{Whammy: an event with a powerful and unpleasant effect; a blow.}.

I wish I could say I'm handling all of this with grace and dignity, but let's be real: I'm a hot mess. And right now, all I want is for Brooke to feel safe and happy, even if it means sacrificing my own sanity in the process.

But hey, at least I've got you, Max. You're the one constant in this whirlwind of chaos, and for that, I'm eternally grateful. Now, if you'll excuse me, I've got some mobile app development to tackle - because apparently, that's still a thing in my life. Wish me luck, Diary. Yoba knows I'm gonna need it.

See you around... Max!

%================================================================================
\sectionNoNumber{Apr 08, 2024 - Andréa and I Kissed}

Ah, Max, strap in for a wild ride 'cause today's entry is a roller-coaster from start to finish. Seriously, grab the popcorn and don't forget the butter - you're gonna need it.

So, let's rewind to the ungodly hour of 2 am. Yours truly was wide awake, thanks to a little thing called insomnia. I blame it on a potent mix of late-night chats with Brooke, a heart-to-heart with you, Max, and a healthy dose of existential dread, combine with Helena lurking on my heart with a pitch fork. Fun times, right?

But just when I was about to hit rock bottom, guess who decided to waltz right into my thoughts like a charming little devil? That's right - Max. Because apparently, my brain thought it'd be a fantastic idea to text her at 4 am. Smooth move, Norman, real smooth.

I was teetering on the edge of desperation and dignity, trying to gauge whether Max actually wanted to talk to me or if she was just being polite. I mean, I've got it bad for her, Max - like, really bad. So, I hit her up with the most pathetic excuse for a text known to mankind: \textit{"Hey... look... when you said you weren't much of a talker, you meant that you didn't want/like to talk, right? As in, not with me, right?"} Pathetic? Maybe. But hey, desperate times call for desperate measures.

Now, before Max could even grace me with a reply, I decided to hit the gym. Because apparently, my brilliant plan to win over the ladies involves bulking up like a wannabe bodybuilder whilst possessing the hair AND facial expressions of Seu Beiçila, from 'A Grande Família' (That joke from Carlos and Myh still hurts me. I am weak like that). Yeah, I'm cringing just thinking and writing about it.

But here's where things take a sharp left turn into Awkwardville. Mid-workout, who decides to grace me with her presence? None other than the ex herself, Helena. Talk about bad timing. She strutted in like she owned the place, all smiles and sunshine, while I'm over here mentally counting down the seconds.

Don't get me wrong, I wasn't rude or anything - just silently cursing my luck and trying to focus on anything but the mental image of Helena frolicking in waterfalls with her suspiciously close "friends." Yeah, let's just say those pictures are burned into my brain like a really bad tattoo.

Okay, so picture this: Ayas, our instructor slash friend slash resident goat enthusiast, surprisingly wasn't on his usual mission to ruffle feathers today. Maybe he finally got the memo that his relentless flirting was about as welcome as a toothache. Then again, knowing Ayas, he's probably just biding his time until he can make another cringe-worthy move. Case in point: his attempt to woo Helena with tales of prostate exams. Yeah, I don't even wanna know.

Speaking of Helena, remember how she joked about telling Ayas we were still together just to get him off our backs? Well, turns out she might've actually gone and done it. Cue the internal panic mode. But hey, who needs a peaceful gym session when you can have an awkward encounter with your ex, right?

So, there I was, trying to dodge Helena's probing questions about my weekend like a ninja, while secretly nursing my wounded pride over the failed Maxine mission. I mean, seriously, Max, how do you even begin to explain to your ex that the weekend you'd been fantasizing about turned into a train wreck of epic proportions?

Helena, bless her heart, decided to give me the lowdown on her own weekend shenanigans - conveniently speeding past the part where she's living it up in waterfalls with her so-called "friends." Yeah, cue the eye roll. But hey, who am I to judge?

Fast forward to breakfast at Helena's place - 'cause apparently, that's a thing. I was all set to make a swift exit after handing over some creatine, but free food is free food, am I right? We chatted a bit, and I couldn't shake the feeling that she was onto me - onto the whole Maxine/Brooke situation, I mean.

But here's where things take a turn for the awkward. As I'm innocently sipping on my scalding hot chocolate (seriously, Max, why is everything out to burn me today?), Helena decides to make a move. Like, a full-on, uninvited attempt at stealing a kiss. Um, excuse me? Last time I checked, we were firmly in the ex-zone, and I had zero interest in revisiting that territory.

Needless to say, I was not amused. I mean, sure, we've had our moments post-breakup, but after everything that's gone down lately? Yeah, hard pass. I managed to dodge her surprise kiss attempt while she still got a smooch, but not before shooting her a look that screamed "What the duck, mate?" And just like that, she was off to her bedroom like nothing happened.

But the awkwardness didn't end there, Max. Oh no, it was just getting started. As I awkwardly tried to avoid making eye contact while Helena undressed right in front of me - seriously, can't a guy catch a break? - the tension in the room reached critical levels. It was like a scene straight out of a bad rom-com, complete with uncomfortable glances from her to me and way too much skin on display, while I gazed upon the beautiful ceiling of her house.

And then, just when I thought things couldn't get any weirder, Helena suddenly seemed... sad. Like, really sad. And here I am, left wondering if it's because I didn't reciprocate her kiss, or if there's something else going on behind the scenes.

And, Max. You're not gonna believe the next plot twist in the ongoing saga of Norman's life. So, remember my little drama earlier about texting Max and getting shot down faster than a clay pigeon? Yeah, turns out it wasn't much of an exaggeration after all. Brace yourself for this one, Max.

After my desperate attempt at fishing for attention - aka, the cringiest message known to man - Max hits me back with a double whammy of truth bombs. First up: \textit{"no, i meant that i just don't like to text on whatsapp in general."} Ouch. Talk about a reality check. And just when I thought things couldn't get any worse, she drops the bombshell: \textit{"yesterday, for example, i was at my boyfriend's house so i didn't use my phone much."} Boom. There it is, in all its painful glory.

Max already has someone. And here I am, the clueless protagonist in my very own rom-com, left to pick up the pieces of my shattered hope. But you know what? It's okay, Max. Really, it is. We didn't have anything to begin with. And besides, considering that she's got her own Chloe already, I might as well embrace my inner Warren and settle into my role as the lovable nerd sidekick, right? I'm pretty sure I'd be more hurt if it wad Brooke.

But enough dwelling on unrequited crushes and missed opportunities. Before I bid you adieu, Max, I need to tell you this: I really could use one of Brooke's legendary hugs right about now. Because let's face it, nothing soothes the soul quite like a warm embrace from her. I don't know about you, Max, but I'm starting to think I need a break from the drama. I know this is just the first part. I am still going to UAST to meet Brooke. Don't fail me now, heart!

Max, buckle up because today took an expected turn - a turn that I'm still trying to wrap my head around. Brace yourself, 'cause here's the scoop: Brooke and I kissed. Yeah, you heard me right. Brooke and I shared a kiss, and I'm still reeling from the aftershocks.

Let's rewind to the beginning of the day, shall we? I rolled into UAST, scarfed down some dinner, and then proceeded to shave off my so-called beard right then and there. Talk about a multitasking marvel, am I right? But I digress.

I bumped into Brooke by the stairs, and from there, things escalated faster than you can say "awkward encounter." We hung out, waiting for her elusive professor - who, shocker, couldn't be bothered to spare her the time of day. Classic academia, am I right?

Anyway, back to the main event. As we strolled along, we ended up on a bench, chatting about everything under the sun - her ex, my near-kiss with Helena this morning, you name it. The atmosphere was charged with tension, and before I knew it, our faces were inches apart. One thing led to another, and suddenly, I found myself kissing her cheek - and she kissed mine back.

But then, in a moment that felt like it was straight out of a cheesy romance novel, our lips met. And let me tell you, Max, it was like fireworks exploding in the night sky - exhilarating, electrifying, and utterly unforgettable. But amidst the rush of emotions, there was a pang of guilt gnawing at my heart, like I was betraying Helena in some way. Stupid, I know. But there it was, lurking in the shadows.

Still, despite the internal turmoil, there was no denying the magic of the moment. We gazed into each other's eyes, giggling like a couple of lovesick teenagers as we dissected the nuances of our impromptu smooch. And let me tell you, feeling her body pressed against mine was nothing short of euphoric.

But all good things must come to an end, and eventually, we had to part ways and catch our respective buses home. Brooke seemed unusually quiet on the ride back, but even in silence, she radiated a quiet beauty that left me speechless.

After the unexpected kiss with Brooke and the ensuing bus ride home, I found myself at Helena's house, reluctantly retrieving my bike. I wish I could say it was a simple exchange, but alas, life rarely grants us such luxuries. We ended up having a conversation - awkward doesn't even begin to cover it. Eventually, I mustered up the courage to lay it all out on the table: it was over between us, for good this time. No turning back, no second chances. She asked for one last kiss, but I couldn't bring myself to oblige. Seeing her cry tore me apart, Max. I never wanted to hurt her, but sometimes, the hardest choices are the ones we have to make.

And then, as if the day hadn't thrown enough curveballs my way, both Brooke and Helena reached out to me. Helena, in particular, had me worried sick - she wanted to throw up when I left her place, and I couldn't shake the feeling that something was seriously wrong. But our conversation was brief, cut short by what she claimed was her mother calling. I couldn't shake the feeling that there was more to the story, but I guess I'll never know for sure.

Next up, I had a heart-to-heart with my mom, breaking the news about Helena and me. And you know what? She took it surprisingly well, almost like she'd seen it coming from a mile away. In fact, she seemed downright pleased, which caught me off guard in the best possible way.

But enough about the drama - let's talk about Brooke. There she was, asleep at her desk, buried under a mountain of textbooks and study materials. And in that moment, Max, I couldn't help but feel a pang of sadness that I hadn't been able to finish that damn Mobile App project. But you know what? It's okay. Some battles are meant to be fought another day.

And now, as the clock ticks closer to midnight, I find myself drifting off to sleep, grateful for your company and the solace you provide. So, until next time, Max, thanks for being here with me through thick and thin. Talk to you later, my friend.

\sectionNoNumber{Apr 09, 2024 - Helena and Her Knight}

Hey there, Max. Brace yourself, 'cause today's entry comes with a big ol' dose of self-loathing. Here goes nothing.

So, Helena had a rough night -- like, hospital-worthy rough. And you know what? I wanted to be there for her, to lend a helping hand and all that jazz. But in the end, I chickened out. I know, I know – I'm a real piece of work.

Instead, she called up Diego – yeah, the same Diego I used to be jealous of – and he swooped in like a knight in shining armor, whisking her off to the hospital in his fancy car. Talk about a punch in the gut.

Fast forward to today, and I find myself paying Helena a visit to check up on her. And let me tell you, Max, what I saw when I walked through that door knocked the wind right out of me. She wasn't wearing anything on her chest, and suddenly, the world felt like it was tilting on its axis.

But here's the kicker, Max: Helena has feelings for Diego. Romantic feelings. And you know what's even worse? I'm pretty damn sure he feels the same way about her. Cue the existential crisis.

When Helena spilled the beans about calling Diego and him rushing to her side like some kind of hero, it felt like a dagger to the heart. And then she dropped the bombshell about seeing the picture of me and Brooke – holding hands, no less – and let me tell you, Max, the guilt hit me like a ton of bricks.

But here's where things take a turn for the downright disastrous: while lying there with Helena, something snapped. Something stupid and reckless and unforgivable. We kissed. And now? Now I'm stuck in this never-ending cycle of self-loathing, wondering why the hell I did something so monumentally stupid.

I wish I could tell you more, Max – spill all the gory details and get it off my chest. But alas, duty calls, and I've got a date with destiny – or rather, a date with Brooke. She's the only one here for me now, and I can't help but feel a pang of relief at the thought of escaping this mess, if only for a little while.

But before I sign off, there's one last thing I need to tell you: I made Helena a promise. "Saturday. 4pm. Here," I said. And you know what? She replied with "bring the thing." But between you and me, Max, it's an empty promise – a feeble attempt to ease the ache in my chest.

I'm sorry, Max. I'll catch up with you later. Until then, wish me luck – I'm gonna need it.

%================================================================================

\sectionNoNumber{Apr 19, 2024 - She admitted it}

This entry is being entered on May 5th. I couldn't enter it in the day, Max. I am sorry. You will know why soon enough, though.

On April 19th, 2024, which was a Friday, Helena, my ex, managed to twist her ankle or something, so we ended up agreeing on something else instead of gym. We had this plan to swing by her "garage" (aka her new eyebrow sanctuary) to catch up and, you know, tame these wild brows of mine. Once i got there, i was waiting for what felt like an eternity. And let me tell you, patience is not my strong suit.

Finally, when she rolls in, her mom goes all FBI on us, and we had to go into stealth mode. Like, ninja-level stealth, Max. We just didn't wanna deal with the whole lecture thing, you know?. It's not like we were plotting a heist or anything, we just wanted some quality hang time without getting lectured. Eventually, momma bear leaves us be, and it's eyebrow time.

Helena spills her guts about feeling all kinds of blue -- lonely, inadequate, you name it. I try my best to sprinkle some wisdom, but I'm not sure if I got through to her. After the brow magic, we settle in for a cozy breakfast sesh. Picture this: Helena perched on that door frame thing we used to share and sit on, and me, camped out on the floor like it's a picnic.

Now, here's where things take a turn. We dive into some  heart-to-heart conversation about Brooke and Professor Diego -- and yeah, Helena swears there's nothing going on there between the two of them. But as we chat, we're inching closer, like magnets drawn together. Before I know it, Helena was on my lap, locking eyes with me like it's the most natural thing in the world. And let me tell you, Max, that moment? That's one of those moments I wish I could just forget everything and make them last forever.

So there we were, deep in conversation, and Helena's rocking this tank top, right? And let me tell you, Max, it was hanging off her shoulders like it had a mind of its own. So, being the helpful friend I am, I lean in to fix her yoke, but she's all, "No. Don't put it back!" Next thing I know, my hands slip, and her breasts are out in the open.

Now, here's where things take a turn. Instead of freaking out, I'm just, like, drawn to them. So, yeah, I start giving 'em a little squeeze, and before I can even second-guess myself, Helena's all, "Kiss 'em." And, well, I'm not one to argue with a good suggestion, Max.

Let me tell you, it was like a rush of all the good feels you can imagine. And when I finally pull away, there she is, inches from my face, those eyes practically begging for more. It's like we're back in those early days, Max, when everything was fireworks and butterflies. And before I know it, our lips are locked.

So, there we are, catching our breath after those mind-blowing kisses, right? And then Helena drops this bombshell: "Quer dar uma entradinha?" Now, I know what you're thinking, Max, and believe me, I was thinking the same thing -- we shouldn't go there. We're not together anymore, and the drama surrounding us is enough to fill a soap opera. I told her we couldn't, but then she hits me with, "Mas você QUER entrar rapidinho?" And well, Max, I caved.

Now, here's the kicker -- I swear, I had no intention of taking things to that level. I just wanted to chat, maybe snuggle up a bit, you know? But somehow, we ended up in this tiny room, and Helena's asking if I brought the "thing." And of course, being the prepared gal that I am, I said yes. Spoiler alert: it wasn't a condom, Max, but a bottle of lube. Yeah, talk about awkward.

So, there we are, standing there, and next thing I know, we're broth drooping our paints just enough to expose that perfect ass of hers that I was always in love with. Nothing could compare. When I got inside her, it felt like one of the best feelings ever. Feeling her body pound mine while having her waist and hips in my hand was perfect. I truly wished that, at that moment, I could relive that moment forever. -- it was pure bliss, Max.

In that moment, I swear I could've stayed there forever. It was like all the stars aligned, and nothing else mattered but us. It felt like the best sex we ever had, Max, and believe me, I've got A LOT of memories to compare it to.

So, there we were, caught up in the heat of the moment, right? But deep down, I knew we had to pump the brakes. I told Helena we needed to stop, but she's all, "Go as fast as you can." And let me tell you, Max, it was like an adrenaline rush, taking things to a whole new level.

And then it happened -- I came, just like that. It was like a split-second decision, but in that moment, it felt like the only choice. And even after, I kept going, pounding away like there was no tomorrow. But eventually, reality hit, and we had to call it quits. Helena, she was beaming, Max. Like, I hadn't seen her that happy in ages.

But then, out of nowhere, it hits me -- this overwhelming sense of guilt, like a ton of bricks crashing down on me. I couldn't believe what I'd just done. I mean, Max, we didn't just kiss -- we went all the way. And as much as I tried to brush it off, the truth hit me like a freight train.

I felt like I had betrayed Brooke, like I had crossed a line I promised myself I wouldn't. My legs went weak, and I had to sit down. I've never felt regret like this before, Max. It's like a weight crushing my chest, suffocating me with every breath.

To be honest, Max, my head's spinning right now. It's like my mind's trapped in a fog, struggling to piece together what just went down. I remember blurting out to Helena that I needed to forget her, to move on, to be done with this mess once and for all. But then it happened -- I asked her about Diego, about whether they'd kissed. he hesitated for a split second. She said yes.

Turns out, they'd shared more than just a peck. She spilled the beans about them locking lips in some square with her crew, like it was no big deal. And get this -- it had been going on for a week, maybe even longer. It hit me like a ton of bricks, Max. The agony, the pain, the sheer disgust -- it was suffocating.

I couldn't hold it back, Max. I wanted to scream, to lash out, to tell her how much I despised her in that moment. I told her I came inside her and that she was taking the pill in that very morning.

I couldn't bear to look at her, Max. The sight of her made my skin crawl. So, I did the only thing I could think of -- I hopped on my bike, told her to take those damn pills, even if it was the last thing she ever did for me, shot her a look of pure disgust, and rode off under that cruel sun.

I didn't even make it to the gym, Max. I just headed straight home, drowning in a sea of anger, betrayal, regret, and heartache. It's like everything I thought I knew was ripped away from me in an instant, leaving me stranded in a world of pain.

You want a joke? Here's one for the books. Just two days earlier, on a seemingly innocent Wednesday, she swore up and down that there was nothing between her and Diego. No spark, no connection -- nada. Fast forward to today, and suddenly they've been swapping kisses for over a week. You know who was right all along? Her mom, Brooke, Carlos, even my own jealousy and overthinking. Talk about a punchline.

But wait, it gets even better. At one point, I mustered up the courage to ask her if any of this even phased her. The whole kissing Diego, breaking my heart into a million pieces, lying through her teeth -- did she regret anything? You'll never believe what she said. Brace yourself -- she actually regretted telling me about her and Diego. Yeah, that's it. That's her big regret. Not the pain she caused, not the trust she shattered -- just that I now know the truth. Comedy gold, right?

Max, this is where the day takes a turn, but thankfully, there's a glimmer of light at the end of the tunnel. So, picture this -- I'm feeling like absolute garbage all day long. I had to drag myself to college, sit through an Infra class, the whole nine yards. And let me tell you, Max, it was like dragging myself through quicksand.

But then, after class, I bump into Brooke and her squad in front of "Bloco 02." And honestly, Max, I don't remember much after that. I was there, but my mind was a million miles away.

Once I finally made it home, Brooke and I hopped on a video call. And let me tell you, Max, it was like she could sense something was off. She straight-up asked if I was okay, and try as I might, I couldn't keep it in. I spilled the beans about my little rendezvous with Helena that morning -- well, most of it, at least.

I told her about our plans to hang out, do my eyebrows, all that jazz. And yeah, I fessed up about the kiss, but I couldn't bring myself to tell her about the other stuff. It's like, deep down, I knew it would crush her, Max. But here's the thing -- just opening up to her, hearing her soothing words, it was like a balm to my wounded heart. She knew exactly what to say, Max, like she had this magic touch that could chase away all the darkness.

And in that moment, Max, I wanted nothing more than to reach through the screen and kiss her senseless. I mean, I'm beyond lucky to have her in my corner, Max. She's my rock, my safe haven in a storm of chaos. But here's the kicker -- as Brooke drifted off to sleep, I found myself at peace, thanks to her. It's like her presence alone was enough to chase away all the demons (and the exes) haunting me. And for that, Max, I'll forever be grateful.

\sectionNoNumber{Apr 03, 2025 - Goodbye, Helena.}

Hey there, Max! Whoa, today took a totally unexpected turn. It's been ages since Brooke advised me to block Helena, my ex, from, like, everywhere. And guess what? I finally mustered the courage to block her on Instagram last night. Go me, right? But, of course, the universe had other plans.

So, there I was at the gym today, getting my bike ready to head out, when BAM! Joelma, Helena's adorable dog, literally bumps into me. And before I could even process that, guess who shows up? Yep, Helena herself. Talk about awkward timing.

She said hi and all that. Then, she straight-up asks me if I blocked her on Instagram. I just stood there like a deer in headlights and didn't say a word. Then she starts telling me about accidentaly meeting my old school friend, Victor, a.k.a Vitão. Apparently, he was totally distraught when he saw her eating a pastel. At this point, I was super confused. I mean, was he hungry or something? So I joked, "Was he starving or what?" And then she drops the bomb: "No, he saw me with good company. He saw me having it with someone else. Thus, he was shocked."

Ouch. That hit me right in the feels. It hurt in a way I can't even describe. I broke eye contact, muttered, "I really didn't need to know that," and pedaled away like my life depended on it. As I rode off, I could faintly hear her say, "I am sorry," but it was too late. I was already gone.

After I got home, I had a video chat with Brooke. I spilled everything about the gym encounter with Helena and how I was totally not okay with it. Brooke, being the amazing wise sage she is, understood completely. She reminded me that I should've blocked Helena everywhere a long time ago and that it's pretty clear Helena gets some twisted pleasure from hurting me.

And then, mid-conversation, Helena sent me a sext on WhatsApp. Seriously, Max? I know this sounds bizarre, but I felt this weird need to save the conversation. I trust you, Max, so I thought I'd store it here with you. I might break down parts of it for context, but honestly, what matters is the conversation itself. I'll take screenshots and rewrite it here, just to be sure it's safe.

I get that I shouldn't keep this conversation, but being able to read it again is oddly comforting. It reminds me. It reminds me that she doesn't want me anymore, that she never wanted to fix things. It reminds me that the person she likes and wants, as she blatantly told me, is Diego, her professor. It makes me... accept that she cheated on me. There's no going back after today. I know this, Max, but keeping a record helps.

After this conversation, I blocked Helena… from everywhere. But I guess she'll always linger in my mind with all those memories I can't block, right? Anyway, here's the conversation, Max. I'm sorry for unloading this on you:

(Just FYI, if you pay attention, it looks like I unblocked her and, then, blocked again. That was pretty much me not knowing how blocking people works in Whatsapp. That "Adeus, Helena" was, indeed, my last ever message to her.)

\includepdf{res/screenshots/2024/20240503 (1).jpeg}
\includepdf{res/screenshots/2024/20240503 (2).jpeg}
\includepdf{res/screenshots/2024/20240503 (3).jpeg}
\includepdf{res/screenshots/2024/20240503 (4).jpeg}
\includepdf{res/screenshots/2024/20240503 (5).jpeg}
\includepdf{res/screenshots/2024/20240503 (6).jpeg}
\includepdf{res/screenshots/2024/20240503 (7).jpeg}
\includepdf{res/screenshots/2024/20240503 (8).jpeg}
\includepdf{res/screenshots/2024/20240503 (9).jpeg}
\includepdf{res/screenshots/2024/20240503 (10).jpeg}
\includepdf{res/screenshots/2024/20240503 (11).jpeg}
\includepdf{res/screenshots/2024/20240503 (12).jpeg}
\includepdf{res/screenshots/2024/20240503 (13).jpeg}

%================================================================================



\chapterNoNumber{Brooke}

\chapterNoNumber{Henka \& Hiking}

\chapterNoNumber{Krista}

\chapterNoNumber{Tabacudos}

\chapterNoNumber{Maka}

\chapterNoNumber{Los Tabacudos}

\chapterNoNumber{Mands}

\chapterNoNumber{Many Mistakes}

\chapterNoNumber{Giangiard}

\chapterNoNumber{Going to Taubaté}

	%%■■■■■■■■■■■■■■■■■■■■■■■■■■■■■■■■■■■■■■■■■■■■■■■■■■■■■■■■■■■■■■■■■■■■■■■■■■■■■■■■
\part{2025}

\chapterNoNumber{Taubaté}

\chapterNoNumber{Facada House}

\chapterNoNumber{Sailor}

\chapterNoNumber{Birthday (2025)}

\chapterNoNumber{Geh}

\chapterNoNumber{Sabagôi}

\chapterNoNumber{...}

\chapterNoNumber{Petrolandia Week}

\chapterNoNumber{Rock 'n Wow}

\chapterNoNumber{Petrova}

\chapterNoNumber{Half Measures}

\chapterNoNumber{Jainy}
\include{src/entries/20251122}

\chapterNoNumber{[Unnamed]}
\include{src/entries/20251213}
	%%■■■■■■■■■■■■■■■■■■■■■■■■■■■■■■■■■■■■■■■■■■■■■■■■■■■■■■■■■■■■■■■■■■■■■■■■■■■■■■■■
\part{2026}

\chapterNoNumber{Midnight}

\include{src/entries/20260105}
\include{src/entries/20260106}

\include{src/entries/20260107}
\include{src/entries/20260113}

\chapterNoNumber{Petrolandia 2026}

	
	%date%================================================================================
\part{Notes}

%■■■■■■■■■■■■■■■■■■■■■■■■■■■■■■■■■■■■■■■■■■■■■■■■■■■■■■■■■■■■■■■■■■■■■■■■■■■■■■■■
\chapter{Cooking}

%================================================================================
\section{Sep 30, 2025 - Macarrão Rápido com Calso de Frango}

Pega metade do macarrão, quebre-o e jogue-o dentro da panela de pressão com o que sobrou de frango cozido, jogue de 1/2 a 3/4 de molho de tomate dentro. Jogue um pouco mais de água para o macarrão cozinhar.

Jogue sal.

Deixe pegar pressão na panela por 8 minutos.

%================================================================================
\section{Nov 04, 2025 - Mousse de Maracuja}


Ingreditentes: \\
3 Maracujas bons. Uma lata/pacote de leite condensado. Um pote de creme de leite.

Opcionais: \\
Um pacotinho de gelatina sem gosto

Passos: \\
Remova a poupa do maracuja e coloque em uma peneira.Usando uma culher, exprema a poupa e tire todo o suco da poupa, deixando apenas as sementes. Coloque o suco da poupa no liquidificador. coloque o leite condensado, o creme d eleite e, se quiser, a gelatina sem gosto. Bata bastante e deixe tudo bem homogeneo. Coloque o que foi batido em um recipiente, lacre-o e coloque-o na geladeira.

Extra: \\
Use as semenetes dos três maracujas para fazer suco. Recomendo colocar açucar ou usar pacote de suco de maracujá em pó. Bata bastante no liquidificador.


	
	%================================================================================
\partNoNumber{THE END}\markboth{}{}

Okay... 

If this is the end of this journal, then wow. We actually made it here.

If I’m reading this at the end of everything, then this page is proof that I was here. That I felt things. That I cared. That I tried. 

And I'm sorry to say this, but it also means I probably died and stopped writing it. Either that or I just stopped writing for other reasons hahaha. 

You’ve seen all my eras:

the confused ones,

the hopeful ones,

the heartbroken ones,

the “I swear I’m fine” ones (I was not fine).

You’ve held my thoughts when they didn’t make sense yet.

You’ve kept secrets I didn’t even know how to say out loud.

You’ve watched me grow without judging, interrupting, or telling me to “just move on”.

If future-me is reading this:
Hey. You survived more than you thought you would.

I hope you’re gentler with yourself now.

I hope you still feel deeply.

I hope you still notice small things.

I hope you still write—somewhere, somehow—even if it’s not here.

And if this journal ends because life got louder, happier, fuller... then that’s okay too.
That just means the story kept going off the page.

So yeah.
Thank you for being my witness.
For holding versions of me that no longer exist.
For reminding me that moments matter—even the quiet ones.

I hope this isn’t really goodbye. That it’s just a pause.

Whatever it is, I hope to see you some day.

Take care. I love you!

\quad

"Norman was here!"

\end{document}
%■■■■■■■■■■■■■■■■■■■■■■■■■■■■■■■■■■■■■■■■■■■■■■■■■■■■■■■■■■■■■■■■■■■■■■■■■■■■■■■■
